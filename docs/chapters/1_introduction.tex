\section{Introduction}

The goal of the application is to provide help to people with visual impairments in navigating indoor environments, such as universities, hospitals, or other places where technologies like GPS location prove inaccurate. Very often, finding one's way around enclosed facilities, especially large ones, can be difficult for people with this type of disability. For this reason, the application aims to constantly provide, through voice support, information about the areas where the person is located and the related points of interest.

To achieve the described goal, it was decided to adopt an approach based on Bluetooth Low Energy beacons and QR codes.
After the development of the application, a specific section of the main building of the School of Engineering at the University of Pisa was chosen to conduct the associated tests. A network of beacons (from \textit{Kontakt.io} \cite{kontakt:beacons}), placed at a certain distance from each other, was then created to divide the environment considered for the experiments into regions. Each area, identified by the application by exploiting the RSSI values associated with the beacons' signals, was equipped with QR codes corresponding to the possible points of interest in it. The user, walking within the test area, is updated, via \textit{Text-to-Speech technology} \cite{android:text-to-speech-ref}, about the region he or she is in. In addition, by framing the QR codes he can pinpoint the exact location of a classroom or bathroom, for example (thanks to the \textit{ML Kit library}, provided by Google \cite{android:mlkit:overview}). Moreover, when encountering a curved section, the application can promptly notify the user and offer them the necessary directions. 

The rest of the paper is organized as follows: Section \vref{sec:architecture} presents the system architecture, Section \vref{sec:experimental_results} shows the experimental results and in Section \vref{sec:conclusion} the conclusions are drawn.
