\section{Conclusions}
In conclusion, this paper addresses the challenge of indoor-assisted navigation for individuals who are blind or visually impaired by proposing a hybrid approach utilizing Bluetooth Low Energy (BLE) beacons and QR codes. 

The developed application was tested in a section of the main building at the School of Engineering, University of Pisa, and the results were satisfactory. The application successfully guided users along the designated path, accurately detecting region switches, nearby points of interest, and incoming curves, while demonstrating good responsiveness and accuracy in providing navigation assistance.

However, several limitations need to be taken into account. Ensuring the availability of Bluetooth Low Energy (BLE) beacons and QR codes is crucial, requiring regular maintenance and upkeep. Scaling up the system to cover larger areas necessitates additional configuration and environmental instrumentation. To enhance accuracy, it is important to overcome signal interference and adapt to different indoor environments. Moreover, incorporating advanced localization techniques and introducing features like point-to-point navigation would greatly improve functionality. Exploring alternative data loading methods, such as a cloud-based database solution, could provide real-time updates and seamless synchronization across devices.

In summary, the proposed hybrid approach combining BLE beacons and QR codes has proven to be effective in assisting individuals with visual impairments in navigating indoor environments. The results obtained from testing demonstrate the feasibility and potential of this approach. However, it is important to consider the mentioned issues and the possibility of incorporating additional functionalities to enhance the overall user experience.
\label{sec:conclusion}
